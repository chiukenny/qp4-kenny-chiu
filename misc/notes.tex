\documentclass[10pt]{article}
% header.tex
% this is where you load pacakges, specify custom formats, etc.

\usepackage[margin=1in,footskip=25pt]{geometry} 
% \usepackage{changepage}
\usepackage{amsmath,amsthm,amssymb,amsfonts,bbm}
\usepackage{mathtools}
% enumitem for custom lists
\usepackage{enumitem}
% Load dsfont this to get proper indicator function (bold 1) with \mathds{1}:
\usepackage{dsfont}
\usepackage{centernot}

\usepackage[usenames,dvipsnames]{xcolor}

% set up commenting code (I will use this during marking)
\definecolor{CommentColor}{rgb}{0,.50,.50}
\newcounter{margincounter}
\newcommand{\displaycounter}{{\arabic{margincounter}}}
\newcommand{\incdisplaycounter}{{\stepcounter{margincounter}\arabic{margincounter}}}
\newcommand{\COMMENT}[1]{\textcolor{CommentColor}{$\,^{(\incdisplaycounter)}$}\marginpar{\scriptsize\textcolor{CommentColor}{ {\tiny $(\displaycounter)$} #1}}}

\usepackage{appendix}

% set up graphics
\usepackage{graphicx}
\DeclareGraphicsExtensions{.pdf,.png,.jpg}
\graphicspath{ {fig/} }

\usepackage{fancyhdr}
\pagestyle{fancy}
\setlength{\headheight}{40pt}


\usepackage[numbers,sort]{natbib}

%%%%%%%%%%%%%%%%%%%%%%%%%%%%%%%%%%%%%%%%%%%%%%%%%%%%%%%%%%%%%%%%%%%%%%%%%%%%%%%%%%%%
% most other packages you might use should be loaded before hyperref
%%%%%%%%%%%%%%%%%%%%%%%%%%%%%%%%%%%%%%%%%%%%%%%%%%%%%%%%%%%%%%%%%%%%%%%%%%%%%%%%%%%%

% Set up hyperlinks:
\definecolor{RefColor}{rgb}{0,0,.65}
\usepackage[colorlinks,linkcolor=RefColor,citecolor=RefColor,urlcolor=RefColor]{hyperref}

\usepackage[capitalize]{cleveref}
\crefname{appsec}{Appendix}{Appendices} % you can tell cleveref what to call things
% defs.tex
% this is where you define custom notation, commands, etc.


%%
% full alphabets of different styles
%%

% bf series
\def\bfA{\mathbf{A}}
\def\bfB{\mathbf{B}}
\def\bfC{\mathbf{C}}
\def\bfD{\mathbf{D}}
\def\bfE{\mathbf{E}}
\def\bfF{\mathbf{F}}
\def\bfG{\mathbf{G}}
\def\bfH{\mathbf{H}}
\def\bfI{\mathbf{I}}
\def\bfJ{\mathbf{J}}
\def\bfK{\mathbf{K}}
\def\bfL{\mathbf{L}}
\def\bfM{\mathbf{M}}
\def\bfN{\mathbf{N}}
\def\bfO{\mathbf{O}}
\def\bfP{\mathbf{P}}
\def\bfQ{\mathbf{Q}}
\def\bfR{\mathbf{R}}
\def\bfS{\mathbf{S}}
\def\bfT{\mathbf{T}}
\def\bfU{\mathbf{U}}
\def\bfV{\mathbf{V}}
\def\bfW{\mathbf{W}}
\def\bfX{\mathbf{X}}
\def\bfY{\mathbf{Y}}
\def\bfZ{\mathbf{Z}}

% bb series
\def\bbA{\mathbb{A}}
\def\bbB{\mathbb{B}}
\def\bbC{\mathbb{C}}
\def\bbD{\mathbb{D}}
\def\bbE{\mathbb{E}}
\def\bbF{\mathbb{F}}
\def\bbG{\mathbb{G}}
\def\bbH{\mathbb{H}}
\def\bbI{\mathbb{I}}
\def\bbJ{\mathbb{J}}
\def\bbK{\mathbb{K}}
\def\bbL{\mathbb{L}}
\def\bbM{\mathbb{M}}
\def\bbN{\mathbb{N}}
\def\bbO{\mathbb{O}}
\def\bbP{\mathbb{P}}
\def\bbQ{\mathbb{Q}}
\def\bbR{\mathbb{R}}
\def\bbS{\mathbb{S}}
\def\bbT{\mathbb{T}}
\def\bbU{\mathbb{U}}
\def\bbV{\mathbb{V}}
\def\bbW{\mathbb{W}}
\def\bbX{\mathbb{X}}
\def\bbY{\mathbb{Y}}
\def\bbZ{\mathbb{Z}}

% cal series
\def\calA{\mathcal{A}}
\def\calB{\mathcal{B}}
\def\calC{\mathcal{C}}
\def\calD{\mathcal{D}}
\def\calE{\mathcal{E}}
\def\calF{\mathcal{F}}
\def\calG{\mathcal{G}}
\def\calH{\mathcal{H}}
\def\calI{\mathcal{I}}
\def\calJ{\mathcal{J}}
\def\calK{\mathcal{K}}
\def\calL{\mathcal{L}}
\def\calM{\mathcal{M}}
\def\calN{\mathcal{N}}
\def\calO{\mathcal{O}}
\def\calP{\mathcal{P}}
\def\calQ{\mathcal{Q}}
\def\calR{\mathcal{R}}
\def\calS{\mathcal{S}}
\def\calT{\mathcal{T}}
\def\calU{\mathcal{U}}
\def\calV{\mathcal{V}}
\def\calW{\mathcal{W}}
\def\calX{\mathcal{X}}
\def\calY{\mathcal{Y}}
\def\calZ{\mathcal{Z}}


%%%%%%%%%%%%%%%%%%%%%%%%%%%%%%%%%%%%%%%%%%%%%%%%%%%%%%%%%%
% text short-cuts
\def\iid{i.i.d.\ } %i.i.d.
\def\ie{i.e.\ }
\def\eg{e.g.\ }
\def\Polya{P\'{o}lya\ }
%%%%%%%%%%%%%%%%%%%%%%%%%%%%%%%%%%%%%%%%%%%%%%%%%%%%%%%%%%

%%%%%%%%%%%%%%%%%%%%%%%%%%%%%%%%%%%%%%%%%%%%%%%%%%%%%%%%%%
% quasi-universal probabilistic and mathematical notation
% my preferences (modulo publication conventions, and clashes like random vectors):
%   vectors: bold, lowercase
%   matrices: bold, uppercase
%   operators: blackboard (e.g., \mathbb{E}), uppercase
%   sets, spaces: calligraphic, uppercase
%   random variables: normal font, uppercase
%   deterministic quantities: normal font, lowercase
%%%%%%%%%%%%%%%%%%%%%%%%%%%%%%%%%%%%%%%%%%%%%%%%%%%%%%%%%%

% operators
\def\P{\bbP} %fundamental probability
\def\E{\bbE} %expectation

\newcommand{\Expect}[1]{\E \left{ #1\right}}
% conditional expectation
\DeclarePairedDelimiterX\bigCond[2]{[}{]}{#1 \;\delimsize\vert\; #2}
\newcommand{\conditional}[3][]{\E_{#1}\bigCond*{#2}{#3}}
\def\Law{\mathcal{L}} %law; this is by convention in the literature
\def\indicator{\mathds{1}} % indicator function

% norms
\newcommand{\norm}[1]{\left\lVert #1 \right\rVert}

% binary relations
\def\condind{{\perp\!\!\!\perp}} %independence/conditional independence
\def\equdist{\stackrel{\text{\rm\tiny d}}{=}} %equal in distribution
\def\equas{\stackrel{\text{\rm\tiny a.s.}}{=}} %euqal amost surely
\def\simiid{\sim_{\mbox{\tiny iid}}} %sampled i.i.d

% common vectors and matrices
\def\onevec{\mathbf{1}}
\def\iden{\mathbf{I}} % identity matrix
\def\supp{\text{\rm supp}}

% misc
% floor and ceiling
\DeclarePairedDelimiter{\ceilpair}{\lceil}{\rceil}
\DeclarePairedDelimiter{\floor}{\lfloor}{\rfloor}
\newcommand{\argdot}{{\,\vcenter{\hbox{\tiny$\bullet$}}\,}} %generic argument dot
%%%%%%%%%%%%%%%%%%%%%%%%%%%%%%%%%%%%%%%%%%%%%%%%%%%%%%%%%%

%%%%%%%%%%%%%%%%%%%%%%%%%%%%%%%%%%%%%%%%%%%%%%%%%%%%%%%%%%
%% some distributions
% continuous
\def\UnifDist{\text{\rm Unif}}
\def\BetaDist{\text{\rm Beta}}
\def\ExpDist{\text{\rm Exp}}
\def\GammaDist{\text{\rm Gamma}}
\def\NormalDist{\text{\rm Normal}}


% discrete
\def\BernDist{\text{\rm Bernoulli}}
\def\BinomDist{\text{\rm Binomial}}
\def\PoissonDist{\text{\rm Poisson}}
%%%%%%%%%%%%%%%%%%%%%%%%%%%%%%%%%%%%%%%%%%%%%%%%%%%%%%%%%%

%%%%%%%%%%%%%%%%%%%%%%%%%%%%%%%%%%%%%%%%%%%%%%%%%%%%%%%%%%
% Project-specific notation should go here
% (Because it's at the end of the file, it can overwrite anything that came before.)



%%%%%%%%%%%%%%%%%%%%%%%%%%%%%%%%%%%%%%%%%%%%%%%%%%%%%%%%%%



\begin{document}

\tableofcontents


\newpage


\section{Optimal Iterative Sketching with the Subsampled Randomized Hadamard Transform}

Based on \citep{Lacotte:2020}.
\\

The performance of iterative Hessian sketch (IHS) has only been studied empirically in existing literature. \citet{Lacotte:2020} show that for IHS with random matrices projected via refreshed (\iid) truncated Haar matrices or subsampled randomized Hadamard transform (SRHT), the limiting rate of convergence is expected to be better than that of IHS with Gaussian random projections. Their other theoretical contributions include a closed form optimal (limiting) step size for IHS with Haar sketches, showing that momentum does not improve performance of IHS with refreshed Haar sketches, and an explicit formula for the second inverse moment of Haar sketches.

\subsection{Background}

\subsubsection{Problem and method}

Consider overdetermined least-squares problems of the form
\[
\bfb^* = \argmin_{\bfb\in\bbR^d}\left\{f(\bfb) = \frac{1}{2}\|\bfX\bfb-\bfy\|^2\right\}
\]
where $\bfX\in\bbR^{n\times d}$ is a given data matrix with $n\geq d$ and $\mathrm{rank}(\bfX)=d$ and $\bfy\in\bbR^n$ is a vector of observations. Iterative Hessian sketch is one iterative method for solving the problem where given step sizes $\{\alpha_t\}$ and momentum $\{\beta_t\}$, the solutions are iteratively updated by
\[
\bfb_{t+1} = \bfb_t - \alpha_tH_t^{-1}\nabla f(\bfb_t)+\beta_t(\bfb_t-\bfb_{t-1}) \;.
\]
The matrix $H_t$ is an approximation of the Hessian $H=\bfX^\T\bfX$ and is given by $H_t=\bfX^\T \bfS_t^\T \bfS_t\bfX$ where $\bfS_0,\ldots,\bfS_t,\ldots$ are refreshed (\iid) $m\times n$ sketching (random) matrices with $m\ll n$. The types of sketches discussed by \citet{Lacotte:2020} include
\begin{enumerate}

\item
Gaussian sketches where $(\bfS_t)_{ij}\overset{\text{\iid}}{\sim} N(0,m^{-1})$. Computing the matrix product $\bfS\bfX$ is $O(mnd)$ in general, which is larger than the cost of $O(nd^2)$ for direct method solvers when $m\geq d$.

\item
truncated Haar sketches using Haar matrices $\bfS_t$ where the rows are orthonormal (\todo: other qualifications? Truncated?). Generating the matrix requires $O(nm^2)$ using a Gram-Schmidt procedure which is larger than $O(nd^2)$.

\item
subsampled randomized Hadamard transform where the sketch $\bfS\bfX$ can be obtained in $O(nd\log m)$ time. Like other orthogonal embeddings, the performance tends to be better than random projections with \iid entries.

\end{enumerate}


\subsubsection{Random matrix theory and tools}

Let $\{\bfM_n\}_n$ be a sequence of $n\times n$ Hermitian random matrices. The empirical spectral distribution (\esd) of $\bfM_n$ is the CDF of its eigenvalues $\lambda_1,\ldots,\lambda_n$ given by $F_{\bfM_n}(x)=\frac{1}{n}\sum_{j=1}^n\mathbbm{1}[\lambda_j\leq x]$ for $x\in\bbR$. The eigenvalues are random and so $F_{\bfM_n}$ is also random. The \esd $F_{\bfM_n}$ converges weakly to the limiting spectral distribution (\lsd) of $\bfM_n$ as $n\rightarrow\infty$.
\\

For a probability measure $\mu$ with support on $[0,\infty)$, its Stieltjes transform is defined over the complex space complementary to the support of $\mu$ as
\[
m_\mu(z) = \int \frac{1}{x-z}\mu(dx) \;.
\]
The $S$-transform of $\mu$ is unique under certain conditions and is defined as the solution to the equation
\[
m_\mu\left(\frac{z+1}{zS_\mu(z)}\right) + zS_\mu(z) = 0\;.
\]

The Marchenko-Pastur theorem says that for a $m\times d$ matrix $\bfS$ where $(\bfS)_{ij}\overset{\text{\iid}}{\sim}N(0,m^{-1})$, then as $m,d\rightarrow\infty$ with $\frac{m}{d}\rightarrow\rho\in(0,1)$, $\bfS^\T\bfS$ has \lsd $F_\rho$ with a Stieltjes transform that is the unique solution of a certain fixed point equation and with a density given by
\[
\mu_\rho(x) = \frac{\sqrt{(1+\sqrt{\rho})^2-x)_+(x-(1-\sqrt{\rho})^2)_+}}{2\pi\rho x}
\]
where $y_+=\max\{0,y\}$.


\subsubsection{Other notation}

Define the aspect ratios $\gamma = \lim_{n,d\rightarrow\infty}\frac{d}{n}\in(0,1)$, $\xi = \lim_{n,m\rightarrow\infty}\frac{m}{n}\in(\gamma,1)$ and $\rho_g=\frac{\gamma}{\xi}\in(0,1)$ where subscript $g$ refers to Gaussians and $h$ refers to Haar or Hadamard. For a sequence $\{\bfb_t\}$, denote the error vector $\Delta_t = \bfU^\T\bfX(\bfb_t-\bfb^*)$ where $\bfU$ is the $n\times d$ matrix of left singular vectors of $\bfX$. Note that $\|\Delta_t\|^2 = \|\bfX(\bfb_t-\bfb^*)\|^2$.


\subsection{Sketching with Haar matrices}

Theorem~3.1 (Optimal IHS with Haar sketches): for refreshed Haar matrices $\{\bfS_t\}$, step sizes $\alpha_t=\frac{\theta_{1,h}}{\theta_{2,h}}$ (defined in Lemma~3.2) and momentum parameters $\beta_t=0$, the sequence of error vectors $\{\Delta_t\}$ satisfies
\[
\rho_h = \left(\lim_{n\rightarrow\infty}\frac{\E\|\Delta_t\|^2}{\|\Delta_0\|^2}\right)^\frac{1}{t} = \rho_g\cdot\frac{\xi(1-\xi)}{\gamma^2+\xi-2\xi\gamma} \;.
\]
For any step sizes $\{a_t\}$ and momentum parameters $\{\beta_t\}$,
\[
\rho_h \leq \liminf_{t\rightarrow\infty}\left(\lim_{n\rightarrow\infty}\frac{\E\|\Delta_t\|^2}{\|\Delta_0\|^2}\right)^\frac{1}{t} \;,
\]
i.e., $\rho_h$ is the optimal rate for Haar embeddings.
\\

Theorem~3.1 says that using the optimal parameters (which has closed forms), the rate at any time step $t\geq1$ is given by
\[
\rho_h^t = \lim_{n\rightarrow\infty}\frac{\E\|\Delta_t\|^2}{\|\Delta_0\|^2}
\]
with $\rho_h<\rho_g$. Momentum also does not provide benefits.
\\

Lemma~3.2 (First two inverse moments of Haar sketches): let $\bfS$ be a $m\times n$ Haar matrix, $\bfU$ a $n\times d$ deterministic matrix with orthonormal columns. Then
\begin{align*}
\theta_{1,h} &= \lim_{n\rightarrow\infty} \frac{1}{d} \mathrm{trace}\left(\E\left[(\bfU^\T\bfS^\T\bfS\bfU)^{-1}\right]\right) = \frac{1-\gamma}{\xi-\gamma} \\
\theta_{2,h} &= \lim_{n\rightarrow\infty} \frac{1}{d} \mathrm{trace}\left(\E\left[(\bfU^\T\bfS^\T\bfS\bfU)^{-2}\right]\right) = \frac{(1-\gamma)(\gamma^2+\xi-2\gamma\xi)}{(\xi-\gamma)^3}
\end{align*}
(Note that $\theta_{i,h}$ is the average of the eigenvalues of $\bfU^\T\bfS^\T\bfS\bfU$ to the power of $-i$.)
\\

\citet{Lacotte:2020} show that as the sketch size $m$ increases relative to $n$, the convergence ratio of Haar sketches versus Gaussian projections scales as $\frac{\rho_h}{\rho_g}\approx (1-\xi)$.


\subsection{Sketching with SRHT matrices}

\citet{Lacotte:2020} consider a version of SRHT where the transform $\bfX\mapsto\bfS\bfX$ first randomly permutes the rows of $\bfX$ before applying the classical transform, i.e., $\bfS=\frac{1}{\sqrt{n}}\bfB\bfH_n\bfD\bfP$ where $\bfB$ is a $n\times n$ diagonal matrix of \iid Bernoulli random variables with success probability $\frac{m}{n}$, $\bfD$ is a $n\times n$ diagonal matrix of \iid sign random variables with uniform probability, and $\bfP$ is a $n\times n$ uniformly distributed permutation matrix. $\bfH_n$ is the $n\times n$ Walsh-Hadamard matrix where for $n=2^p$ for $p\geq1$, the matrix is defined recursively as
\[
\bfH_n = \begin{bmatrix}
\bfH_\frac{n}{2} & \bfH_\frac{n}{2} \\
\bfH_\frac{n}{2} & -\bfH_\frac{n}{2}
\end{bmatrix}
\]
with $\bfH_1=1$. Before applying the transformation to $\bfX$, the zero rows of $\bfS$ are discarded and so $\bfS$ is a $M\times n$ orthogonal matrix with $M\sim\mathrm{Binomial}(\frac{m}{n},n)$, and $\frac{M}{n}\rightarrow\xi$ as $n\rightarrow\infty$. Note that $\bfS$ is still referred to as a $m\times n$ SRHT matrix.
\\

Theorem~4.1 (IHS with SRHT sketches). Suppose that $\bfb_0$ is random and that the error vector $\Delta_0$ satisfies $\E\left[\Delta_0\Delta_0^\T\right]=d^{-1}\bfI_d$. Then for refreshed SRHT matrices $\{\bfS_t\}$, step sizes $\alpha_t=\frac{\theta_{1,h}}{\theta_{2,h}}$ and momentum parameters $\beta_t=0$,  the sequence $\{\Delta_t\}$ satisfies
\[
\rho_s = \left(\lim_{n\rightarrow\infty}\frac{\E\left[\|\Delta_t\|^2\right]}{\E\left[\|\Delta_0\|^2\right]}\right)^\frac{1}{t} = \rho_g \cdot \frac{\xi(1-\xi)}{\gamma^2+\xi-2\xi\gamma} = \rho_h \;.
\]
\todo


\newpage

\bibliographystyle{plainnat}
\bibliography{../references/qp}

\end{document}
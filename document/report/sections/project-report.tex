% !TEX root = ../main.tex

% Project report section

\section{Project report}

\subsection{Introduction}

Ridge regression is a special case of regularized least squares where the penalty function is chosen to be the $\ell_2$-norm of the model parameters. Given data matrix $\bfX\in\bbR^{n\times d}$, observations $\bfy\in\bbR^n$ and a regularization parameter $\lambda>0$, ridge regression obtains estimates of the parameters as the solution to the optimization problem
\[
\bfb^* = \argmin_{\bfb\in\bbR^d} \frac{1}{2}\|\bfX\bfb-\bfy\|_2^2 + \frac{\lambda}{2}\|\bfb\|_2^2 \;.
\]
While ridge regression can be motivated as a method for reducing overfitting in ordinary least squares (OLS), it also has its computational and analytical benefits over OLS. When $\bfX$ does not have full column rank (e.g., when $n< d$), then $\bfX^\T\bfX$ is singular and the OLS solution is non-unique. When $\bfX$ is full rank but ill-conditioned, then small changes in $\bfX$ lead to large changes in $(\bfX^\T\bfX)^{-1}$ and consequently in the OLS solution. Ridge regression addresses both of these issues by minimizing the variance and mean squared error at the cost of introducing a small bias~\citep{Chowdhury:2018}. The ridge regression solution is unique and is given by
\[
\bfb^* = \left(\bfX^\T\bfX + \lambda\bfI_d\right)^{-1}\bfX^\T\bfy \;.
\]
In this report, we analyze the theoretical properties of a partial Newton sketch algorithm~\citep{Pilanci:2017} as an iterative solver for the ridge regression problem. In particular, we attempt to derive an optimal convergence rate and an optimal step size following the approach of \citet{Lacotte:2020} for iterative Hessian sketch with OLS using asymptotic results from random matrix theory and free probability. We show that while ridge regression can be considered a simple extension to OLS, extending the analysis approach of \citet{Lacotte:2020} to partial Newton sketch is not trivial. \todo
\\

This report is organized as follows: Section~\ref{sec:background} provides background about sketching and describes the OLS results by \citet{Lacotte:2020} that we aim to extend to ridge regression; Section~\ref{sec:literature} highlights relevant work in the literature; Section~\ref{sec:theory} discusses our attempts to analyze Newton sketch for ridge regression and the key differences from OLS that makes the problem challenging; Section~\ref{sec:empirical} describes \todo; and Section~\ref{sec:conclusion} summarizes our findings and concludes this report.

\subsection{Background} \label{sec:background}

This section provides additional background about Newton and iterative Hessian sketch. \todo

\subsubsection{Sketching and Newton sketch}

\subsubsection{Random matrix theory and free probability}

\subsubsection{Notation}

Define $\Delta_t=\bfU^\T\bfX\left(\bfb_t-\bfb^*\right)$.

\subsection{Related work} \label{sec:literature}

\citet{Chowdhury:2018}

\citet{Wang:2017}

\citet{Cohen:2017}

\citet{Lacotte:2020}

\subsection{Newton sketch for ridge regression} \label{sec:ridgesketch}

Consider the ridge regression loss function
\[
f(\bfb) = \frac{1}{2}\|\bfX\bfb-\bfy\|_2^2 + \frac{\lambda}{2}\|\bfb\|_2^2
\]
for $\bfb\in\bbR^d$ given data matrix $\bfX\in\bbR^{n\times d}$ with $d\ll n$, responses $\bfy\in\bbR^n$ and a regularization parameter $\lambda>0$. The gradient and Hessian of the function are respectively given by
\begin{align*}
\nabla f(\bfb) &= \left(\bfX^\T\bfX+\lambda\bfI_d\right)\bfb - \bfX^\T\bfy \;, \\
\bfH = \nabla^2f(\bfb) &= \bfX^\T\bfX+\lambda\bfI_d \;.
\end{align*}
An iterative solver based on Newton's method for minimizing the loss function computes the updates
\begin{align*}
\bfb_{t+1} &= \bfb_t - \alpha_t\bfH^{-1}\nabla f(\bfb_t) \\
&= \bfb_t - \alpha_t\left(\bfX^\T\bfX+\lambda\bfI_d\right)^{-1}\left(\left(\bfX^\T\bfX+\lambda\bfI_d\right)\bfb_t-\bfX^\T\bfy\right) \;.
\end{align*}
We consider a partial Newton sketch variant of this update that approximates the Hessian by the sketched version
\[
\bfH_t = \bfX^\T\bfS_t^\T\bfS_t\bfX + \lambda\bfI_d
\]
where $\bfS_t$ is a $m\times n$ sketching matrix, $m\ll n$, that is resampled every iteration. \todo:\citet{Chowdhury:2018} appendix results?. The partial Newton sketch updates for ridge regression are then given by
\[
\bfb_{t+1} = \bfb_t - \alpha_t\left(\bfX^\T\bfS_t^\T\bfS_t\bfX + \lambda\bfI_d\right)^{-1}\left(\left(\bfX^\T\bfX+\lambda\bfI_d\right)\bfb_t-\bfX^\T\bfy\right) \;.
\]
Note that \citet{Chowdhury:2018} and \citet{Wang:2017}) also considered this sketched update for ridge regression. However, our analysis of this method differs from theirs in that we adopt the asymptotic random matrix theoretic approach from \citet{Lacotte:2020}. Also note that we do not consider updates with momentum as \citet{Lacotte:2020} did as we show that extending their analysis approach from OLS to ridge regression is already non-trivial.



\subsection{Analysis attempt based on random matrix theory} \label{sec:theory}

In this section, we show that the proof technique used to obtain Theorems~3.1 and 4.1 of \citep{Lacotte:2020} do not easily generalize to the partial Newton sketch updates for ridge regression. We follow the general procedure of the proofs and show how far we can get with the ridge regression setup. We also highlight the key differences between OLS and ridge regression that leads to problems in the proof and discuss possible solutions for rectifying these problems in future work.
\\

The following conjecture formalizes the result analogous to Theorems~3.1 and 4.1 that we would like to prove. Note that additional assumptions will be added to the conjecture as we progress through the proof.

\begin{conjecture} \label{con:ridge}
Consider the partial Newton update for ridge regression described in Section~\ref{sec:ridgesketch}. For some optimal step size $\alpha_t$, the sequence of error vectors $\{\Delta_t\}$ satisfies
\[
\rho = \left(\lim_{n\rightarrow\infty}\frac{\E\left[\|\Delta_t\|_2^2\right]}{\|\Delta_0\|_2^2}\right)^\frac{1}{t}
\]
where $\rho$ is the rate of convergence with some closed-form expression.
\end{conjecture}

We begin our attempt to prove Conjecture~\ref{con:ridge} following the proofs by \citet{Lacotte:2020}. Using the fact that the ridge regression solution satisfies the equation
\[
(\bfX^\T\bfX + \lambda\bfI_d)\bfb^* = \bfX^\T\bfY \;,
\]
the update can be rewritten as
\begin{align*}
\bfb_{t+1} &= \bfb_t - \alpha_t\left(\bfX^\T\bfS_t^\T\bfS_t\bfX + \lambda\bfI_d\right)^{-1}\left(\left(\bfX^\T\bfX+\lambda\bfI_d\right)\bfb_t-\left(\bfX^\T\bfX+\lambda\bfI_d\right)\bfb^*\right) \\
&= \bfb_t - \alpha_t\left(\bfX^\T\bfS_t^\T\bfS_t\bfX + \lambda\bfI_d\right)^{-1}\left(\bfX^\T\bfX+\lambda\bfI_d\right)\left(\bfb_t-\bfb^*\right) \;.
\end{align*}
Let $\bfX = \bfU\Sigma\bfV^\T$ be the thin singular value decomposition of $\bfX$ where $\bfU$ is a $n\times d$ semi-orthogonal matrix, $\bfV$ is a $d\times d$ orthogonal matrix, and $\Sigma$ is a $d\times d$ diagonal matrix with the singular values of $\bfX$ on the diagonal. Then we can write
\begin{align*}
\bfX^\T\bfX+\lambda\bfI_d &= \bfV\Sigma^2\bfV^\T + \lambda \bfV\bfV^\T \\
&= \bfV\left(\Sigma^2+ \lambda\bfI_d\right)\bfV^\T \;, \\
\left(\bfX^\T\bfS_t^\T\bfS_t\bfX + \lambda\bfI_d\right)^{-1} &= \left(\bfV\Sigma\bfU^\T\bfS_t^\T\bfS_t\bfU\Sigma\bfV^\T + \lambda\bfV\bfV^\T\right)^{-1} \\
&= \bfV\left(\Sigma\bfU^\T\bfS_t^\T\bfS_t\bfU\Sigma + \lambda\bfI_d\right)^{-1}\bfV^\T \;.
\end{align*}
However, in order to later on obtain an expression in terms of $\Delta_t$ as in the original proof, we require that the data matrix be full column rank. This is a less than ideal assumption to make as one of the advantages of ridge regression is being able to obtain an unique solution with non-full rank data matrices. We return to this point in a later discussion how we may avoid this assumption. \todo

\begin{assumption} \label{asp:rank}
The data matrix $\bfX$ has full column rank.
\end{assumption}

Under Assumption~\ref{asp:rank}, the singular values of $\bfX$ are non-zero and so the above matrices can be rewritten as
\begin{align*}
\bfX^\T\bfX+\lambda\bfI_d &= \bfV\Sigma\left(\bfI_d + \lambda\Sigma^{-2}\right)\Sigma\bfV^\T \;, \\
\left(\bfX^\T\bfS_t^\T\bfS_t\bfX + \lambda\bfI_d\right)^{-1} &= \bfV\Sigma^{-1}\left(\bfU^\T\bfS_t^\T\bfS_t\bfU + \lambda\Sigma^{-2}\right)^{-1}\Sigma^{-1}\bfV^\T \;.
\end{align*}
Replacing the corresponding matrices in the update with these identities gives
\[
\bfb_{t+1} = \bfb_t - \alpha_t\bfV\Sigma^{-1}\left(\bfU^\T\bfS_t^\T\bfS_t\bfU + \lambda\Sigma^{-2}\right)^{-1}\left(\bfI_d + \lambda\Sigma^{-2} \right)\Sigma\bfV^\T\left(\bfb_t-\bfb^*\right) \;.
\]
Multiplying both sides by $\bfU^\T\bfX$ gives
\begin{align*}
\bfU^\T\bfX\bfb_{t+1} &= \bfU^\T\bfX\bfb_t - \alpha_t\bfU^\T\bfX\bfV\Sigma^{-1}\left(\bfU^\T\bfS_t^\T\bfS_t\bfU + \lambda\Sigma^{-2}\right)^{-1}\left(\bfI_d + \lambda\Sigma^{-2} \right)\Sigma\bfV^\T\left(\bfb_t-\bfb^*\right) \\
&= \bfU^\T\bfX\bfb_t - \alpha_t\left(\bfU^\T\bfS_t^\T\bfS_t\bfU + \lambda\Sigma^{-2}\right)^{-1}\left(\bfI_d + \lambda\Sigma^{-2} \right)\Sigma\bfV^\T\left(\bfb_t-\bfb^*\right)
\end{align*}
and then subtracting both sides by $\bfU^\T\bfX\bfb^*$ gives
\begin{align*}
\bfU^\T\bfX(\bfb_{t+1}-\bfb^*) &= \bfU^\T\bfX(\bfb_t-\bfb^*) - \alpha_t\left(\bfU^\T\bfS_t^\T\bfS_t\bfU + \lambda\Sigma^{-2}\right)^{-1}\left(\bfI_d + \lambda\Sigma^{-2} \right)\Sigma\bfV^\T\left(\bfb_t-\bfb^*\right) \\
&= \left(\bfI_d - \alpha_t\left(\bfU^\T\bfS_t^\T\bfS_t\bfU + \lambda\Sigma^{-2}\right)^{-1}\left(\bfI_d + \lambda\Sigma^{-2}\right)\right)\bfU^\T\bfX\left(\bfb_t-\bfb^*\right) \;.
\end{align*}
Let $\bfQ_t=\bfI_d - \alpha_t\left(\bfU^\T\bfS_t^\T\bfS_t\bfU + \lambda\Sigma^{-2}\right)^{-1}\left(\bfI_d + \lambda\Sigma^{-2}\right)$. Therefore by definition, we have $\Delta_{t+1} = \bfQ_t\Delta_t$ and
\[
\|\Delta_{t+1}\|^2 = \Delta_t^\T\bfQ_t^\T\bfQ_t\Delta_t \;.
\]
Taking the expectation with respect to $\bfS_t$, we get
\[
\E\left[\|\Delta_{t+1}\|^2\right] = \Delta_t^\T\E\left[\bfQ_t^\T\bfQ_t\right]\Delta_t
\]
where
\begin{align*}
\E\left[\bfQ_t^\T\bfQ_t\right] &= \bfI_d - \alpha_t\E\left[\left(\bfU^\T\bfS_t^\T\bfS_t\bfU + \lambda\Sigma^{-2}\right)^{-1}\right]\left(\bfI_d + \lambda\Sigma^{-2}\right) \\
&\quad - \alpha_t\left(\bfI_d + \lambda\Sigma^{-2}\right)\E\left[\left(\bfU^\T\bfS_t^\T\bfS_t\bfU + \lambda\Sigma^{-2}\right)^{-1}\right] \\
&\quad + \alpha_t^2\left(\bfI_d + \lambda\Sigma^{-2}\right)\E\left[\left(\bfU^\T\bfS_t^\T\bfS_t\bfU + \lambda\Sigma^{-2}\right)^{-2}\right]\left(\bfI_d + \lambda\Sigma^{-2}\right) \;.
\end{align*}
At this point, we run into our first major obstacle that prevents us from applying the key step of the proof of Theorem~3.1. In Theorem~3.1 for OLS, the expression that is obtained from taking the expectation is
\[
\E\left[\|\Delta_{t+1}\|^2\right] = \Delta_t^\T\E\left[\bfR_t^2\right]\Delta_t
\]
where
\[
\E\left[\bfR_t^2\right] = \bfI_d - 2\alpha_t\E\left[\left(\bfU^\T\bfS_t^\T\bfS_t\bfU\right)^{-1}\right] + \alpha_t^2\E\left[\left(\bfU^\T\bfS_t^\T\bfS_t\bfU\right)^{-2}\right] \;.
\]
The proof of Theorem~3.1 proceeds to recognize that the matrix $\bfS_t\bfU$ can be embedded into a Haar matrix and is therefore rotationally invariant. Using exchangeability arguments, the expectations $\E\left[\left(\bfU^\T\bfS_t^\T\bfS_t\bfU\right)^{-p}\right]$ have a simple closed-form expression in terms of the inverse moments from which the rest of the proof follows. We do not have rotational invariance in our ridge regression case, and so we follow the proof of Theorem~4.1 from this point onwards. We require an additional assumption on the initialization of the problem.

\begin{assumption} \label{asp:initialization}
The initial error vector $\Delta_0$ is random and satisfies $\E\left[\Delta_0\Delta_0^\T\right]=\frac{\bfI_d}{d}$.\footnote{The convergence rate $\rho$ in the statement of Conjecture~\ref{con:ridge} is also redefined as $\left(\lim_{n\rightarrow\infty}\frac{\E\left[\|\Delta_t\|_2^2\right]}{\E\left[\|\Delta_0\|_2^2\right]}\right)^\frac{1}{t}$.}
\end{assumption}

Under Assumption~\ref{asp:initialization}, taking the expectation with respect to $\bfS_t$ instead gives
\begin{align*}
\E\left[\|\Delta_{t+1}\|^2\right] &= \E\left[\Delta_t^\T\bfQ_t^\T\bfQ_t\Delta_t\right] \\
&= \E\left[\Delta_0^\T\bfQ_0^\T\ldots \bfQ_t^\T\bfQ_t\ldots\bfQ_0\Delta_0\right] \\
&= \E\left[\mathrm{trace}\left(\Delta_0^\T\bfQ_0^\T\ldots \bfQ_t^\T\bfQ_t\ldots\bfQ_0\Delta_0\right)\right] \\
&= \mathrm{trace}\left(\E\left[\bfQ_0^\T\ldots \bfQ_t^\T\bfQ_t\ldots\bfQ_0\Delta_0\Delta_0^\T\right]\right)
\end{align*}
\todo: are $\bfQ$ and rank one matrix $\Delta_0\Delta_0^\T$ free?

\todo satisfies conditions (limiting spectral distribution as $n\rightarrow\infty$ and others?). Then we have $\Delta_{t+1}=\bfQ_t\Delta_t$ and so
\begin{align*}
\E\left[\|\Delta_{t+1}\|^2\right] &= \mathrm{trace}\left(\E\left[\Delta_0^\T\bfQ_0\ldots \bfQ_{t-1}\bfQ_{t-1}\ldots\bfQ_0\Delta_0\right]\right) \\
&= \mathrm{trace}\left(\E\left[\bfQ_0\ldots \bfQ_{t-1}\bfQ_{t-1}\ldots\bfQ_0\Delta_0\Delta_0^\T\right]\right) \\
&= \mathrm{trace}\left(\E\left[\bfQ_1\ldots \bfQ_{t-1}\bfQ_{t-1}\ldots\bfQ_1\bfQ_0^2\right]\E\left[\Delta_0\Delta_0^\T\right]\right) \\
&= \frac{1}{d}\mathrm{trace}\left(\E\left[\bfQ_1\ldots \bfQ_{t-1}\bfQ_{t-1}\ldots\bfQ_1\bfQ_0^2\right]\right)
\end{align*}
using the independence of $\Delta_0$ and $\bfQ_i$ and using Assumption~2. Then taking the limit in $n$ and recursively applying the fact that $\bfQ_0^2$ is asymptotically free from $\bfQ_{t-1}\ldots\bfQ_1$ \todo, we get
\begin{align*}
\lim_{n\rightarrow\infty}\E\left[\|\Delta_{t+1}\|^2\right] &= \lim_{n\rightarrow\infty}\frac{1}{d}\mathrm{trace}\left(\E\left[\bfQ_1\ldots \bfQ_{t-1}\bfQ_{t-1}\ldots\bfQ_1\bfQ_0^2\right]\right) \\
&= \lim_{n\rightarrow\infty}\frac{1}{d}\mathrm{trace}\left(\E\left[\bfQ_0^2\right]\right)\lim_{n\rightarrow\infty}\frac{1}{d}\mathrm{trace}\left(\E\left[\bfQ_2\ldots \bfQ_{t-1}\bfQ_{t-1}\ldots\bfQ_2\bfQ_1^2\right]\right) \\
&= \prod_{i=0}^{t-1}\lim_{n\rightarrow\infty}\frac{1}{d}\mathrm{trace}\left(\E\left[\bfQ_i^2\right]\right)
\end{align*}
\todo The expectation of $\bfQ_i$ is given by
\begin{align*}
\E\left[\bfQ_i^2\right] &= \bfI_d - \alpha_i\E\left[\left(\bfU^\T\bfS_i^\T\bfS_i\bfU + \lambda\Sigma^{-2}\right)^{-1}\right]\left(\bfI_d + \lambda\Sigma^{-2}\right) \\
&\quad - \alpha_i\left(\bfI_d + \lambda\Sigma^{-2}\right)\E\left[\left(\bfU^\T\bfS_i^\T\bfS_i\bfU + \lambda\Sigma^{-2}\right)^{-1}\right] \\
&\quad + \alpha_i^2\left(\bfI_d + \lambda\Sigma^{-2}\right)\E\left[\left(\bfU^\T\bfS_i^\T\bfS_i\bfU + \lambda\Sigma^{-2}\right)^{-2}\right]\left(\bfI_d + \lambda\Sigma^{-2}\right)
\end{align*}
and the normalized limiting trace is given by
\begin{align*}
\lim_{n\rightarrow\infty}\frac{1}{d}\mathrm{trace}\left(\E\left[\bfQ_i^2\right]\right) &= 1 - \frac{2\alpha_i}{d}\lim_{n\rightarrow\infty}\mathrm{trace}\left(\E\left[\left(\bfU^\T\bfS_i^\T\bfS_i\bfU + \lambda\Sigma^{-2}\right)^{-1}\right]\left(\bfI_d + \lambda\Sigma^{-2}\right)\right) \\
&\quad + \frac{\alpha_i^2}{d}\lim_{n\rightarrow\infty}\mathrm{trace}\left(\E\left[\left(\bfU^\T\bfS_i^\T\bfS_i\bfU + \lambda\Sigma^{-2}\right)^{-2}\right]\left(\bfI_d + \lambda\Sigma^{-2}\right)^2\right)
\end{align*}

\subsubsection{Full column rank assumption}

\subsection{Empirical experiments} \label{sec:empirical}


\subsection{Conclusion} \label{sec:conclusion}
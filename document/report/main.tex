%%%%%%%%%%%%%%%%%%%%%%%%%%%%%%%%%%%%%%%%%%%%%%%%%%%%%%%%%%%%%%%%%%%%%%%%%%%%%%%%%%%%
% Template for STAT 548 Qualifying Paper Report
% Author: Ben Bloem-Reddy <benbr@stat.ubc.ca>
% Revised: Daniel J. McDonald <daniel@stat.ubc.ca>
% Date: 3 August 2021
%%%%%%%%%%%%%%%%%%%%%%%%%%%%%%%%%%%%%%%%%%%%%%%%%%%%%%%%%%%%%%%%%%%%%%%%%%%%%%%%%%%%

% Note: You will get an empty bibliography warning when compiling until you include a citation.

\documentclass[12pt]{article}
% header.tex
% this is where you load pacakges, specify custom formats, etc.

\usepackage[margin=1in,footskip=25pt]{geometry} 
% \usepackage{changepage}
\usepackage{amsmath,amsthm,amssymb,amsfonts,bbm}
\usepackage{mathtools}
% enumitem for custom lists
\usepackage{enumitem}
% Load dsfont this to get proper indicator function (bold 1) with \mathds{1}:
\usepackage{dsfont}
\usepackage{centernot}

\usepackage[usenames,dvipsnames]{xcolor}

% set up commenting code (I will use this during marking)
\definecolor{CommentColor}{rgb}{0,.50,.50}
\newcounter{margincounter}
\newcommand{\displaycounter}{{\arabic{margincounter}}}
\newcommand{\incdisplaycounter}{{\stepcounter{margincounter}\arabic{margincounter}}}
\newcommand{\COMMENT}[1]{\textcolor{CommentColor}{$\,^{(\incdisplaycounter)}$}\marginpar{\scriptsize\textcolor{CommentColor}{ {\tiny $(\displaycounter)$} #1}}}

\usepackage{appendix}

% set up graphics
\usepackage{graphicx}
\DeclareGraphicsExtensions{.pdf,.png,.jpg}
\graphicspath{ {fig/} }

\usepackage{fancyhdr}
\pagestyle{fancy}
\setlength{\headheight}{40pt}


\usepackage[numbers,sort]{natbib}

%%%%%%%%%%%%%%%%%%%%%%%%%%%%%%%%%%%%%%%%%%%%%%%%%%%%%%%%%%%%%%%%%%%%%%%%%%%%%%%%%%%%
% most other packages you might use should be loaded before hyperref
%%%%%%%%%%%%%%%%%%%%%%%%%%%%%%%%%%%%%%%%%%%%%%%%%%%%%%%%%%%%%%%%%%%%%%%%%%%%%%%%%%%%

% Set up hyperlinks:
\definecolor{RefColor}{rgb}{0,0,.65}
\usepackage[colorlinks,linkcolor=RefColor,citecolor=RefColor,urlcolor=RefColor]{hyperref}

\usepackage[capitalize]{cleveref}
\crefname{appsec}{Appendix}{Appendices} % you can tell cleveref what to call things
% defs.tex
% this is where you define custom notation, commands, etc.


%%
% full alphabets of different styles
%%

% bf series
\def\bfA{\mathbf{A}}
\def\bfB{\mathbf{B}}
\def\bfC{\mathbf{C}}
\def\bfD{\mathbf{D}}
\def\bfE{\mathbf{E}}
\def\bfF{\mathbf{F}}
\def\bfG{\mathbf{G}}
\def\bfH{\mathbf{H}}
\def\bfI{\mathbf{I}}
\def\bfJ{\mathbf{J}}
\def\bfK{\mathbf{K}}
\def\bfL{\mathbf{L}}
\def\bfM{\mathbf{M}}
\def\bfN{\mathbf{N}}
\def\bfO{\mathbf{O}}
\def\bfP{\mathbf{P}}
\def\bfQ{\mathbf{Q}}
\def\bfR{\mathbf{R}}
\def\bfS{\mathbf{S}}
\def\bfT{\mathbf{T}}
\def\bfU{\mathbf{U}}
\def\bfV{\mathbf{V}}
\def\bfW{\mathbf{W}}
\def\bfX{\mathbf{X}}
\def\bfY{\mathbf{Y}}
\def\bfZ{\mathbf{Z}}

% bb series
\def\bbA{\mathbb{A}}
\def\bbB{\mathbb{B}}
\def\bbC{\mathbb{C}}
\def\bbD{\mathbb{D}}
\def\bbE{\mathbb{E}}
\def\bbF{\mathbb{F}}
\def\bbG{\mathbb{G}}
\def\bbH{\mathbb{H}}
\def\bbI{\mathbb{I}}
\def\bbJ{\mathbb{J}}
\def\bbK{\mathbb{K}}
\def\bbL{\mathbb{L}}
\def\bbM{\mathbb{M}}
\def\bbN{\mathbb{N}}
\def\bbO{\mathbb{O}}
\def\bbP{\mathbb{P}}
\def\bbQ{\mathbb{Q}}
\def\bbR{\mathbb{R}}
\def\bbS{\mathbb{S}}
\def\bbT{\mathbb{T}}
\def\bbU{\mathbb{U}}
\def\bbV{\mathbb{V}}
\def\bbW{\mathbb{W}}
\def\bbX{\mathbb{X}}
\def\bbY{\mathbb{Y}}
\def\bbZ{\mathbb{Z}}

% cal series
\def\calA{\mathcal{A}}
\def\calB{\mathcal{B}}
\def\calC{\mathcal{C}}
\def\calD{\mathcal{D}}
\def\calE{\mathcal{E}}
\def\calF{\mathcal{F}}
\def\calG{\mathcal{G}}
\def\calH{\mathcal{H}}
\def\calI{\mathcal{I}}
\def\calJ{\mathcal{J}}
\def\calK{\mathcal{K}}
\def\calL{\mathcal{L}}
\def\calM{\mathcal{M}}
\def\calN{\mathcal{N}}
\def\calO{\mathcal{O}}
\def\calP{\mathcal{P}}
\def\calQ{\mathcal{Q}}
\def\calR{\mathcal{R}}
\def\calS{\mathcal{S}}
\def\calT{\mathcal{T}}
\def\calU{\mathcal{U}}
\def\calV{\mathcal{V}}
\def\calW{\mathcal{W}}
\def\calX{\mathcal{X}}
\def\calY{\mathcal{Y}}
\def\calZ{\mathcal{Z}}


%%%%%%%%%%%%%%%%%%%%%%%%%%%%%%%%%%%%%%%%%%%%%%%%%%%%%%%%%%
% text short-cuts
\def\iid{i.i.d.\ } %i.i.d.
\def\ie{i.e.\ }
\def\eg{e.g.\ }
\def\Polya{P\'{o}lya\ }
%%%%%%%%%%%%%%%%%%%%%%%%%%%%%%%%%%%%%%%%%%%%%%%%%%%%%%%%%%

%%%%%%%%%%%%%%%%%%%%%%%%%%%%%%%%%%%%%%%%%%%%%%%%%%%%%%%%%%
% quasi-universal probabilistic and mathematical notation
% my preferences (modulo publication conventions, and clashes like random vectors):
%   vectors: bold, lowercase
%   matrices: bold, uppercase
%   operators: blackboard (e.g., \mathbb{E}), uppercase
%   sets, spaces: calligraphic, uppercase
%   random variables: normal font, uppercase
%   deterministic quantities: normal font, lowercase
%%%%%%%%%%%%%%%%%%%%%%%%%%%%%%%%%%%%%%%%%%%%%%%%%%%%%%%%%%

% operators
\def\P{\bbP} %fundamental probability
\def\E{\bbE} %expectation

\newcommand{\Expect}[1]{\E \left{ #1\right}}
% conditional expectation
\DeclarePairedDelimiterX\bigCond[2]{[}{]}{#1 \;\delimsize\vert\; #2}
\newcommand{\conditional}[3][]{\E_{#1}\bigCond*{#2}{#3}}
\def\Law{\mathcal{L}} %law; this is by convention in the literature
\def\indicator{\mathds{1}} % indicator function

% norms
\newcommand{\norm}[1]{\left\lVert #1 \right\rVert}

% binary relations
\def\condind{{\perp\!\!\!\perp}} %independence/conditional independence
\def\equdist{\stackrel{\text{\rm\tiny d}}{=}} %equal in distribution
\def\equas{\stackrel{\text{\rm\tiny a.s.}}{=}} %euqal amost surely
\def\simiid{\sim_{\mbox{\tiny iid}}} %sampled i.i.d

% common vectors and matrices
\def\onevec{\mathbf{1}}
\def\iden{\mathbf{I}} % identity matrix
\def\supp{\text{\rm supp}}

% misc
% floor and ceiling
\DeclarePairedDelimiter{\ceilpair}{\lceil}{\rceil}
\DeclarePairedDelimiter{\floor}{\lfloor}{\rfloor}
\newcommand{\argdot}{{\,\vcenter{\hbox{\tiny$\bullet$}}\,}} %generic argument dot
%%%%%%%%%%%%%%%%%%%%%%%%%%%%%%%%%%%%%%%%%%%%%%%%%%%%%%%%%%

%%%%%%%%%%%%%%%%%%%%%%%%%%%%%%%%%%%%%%%%%%%%%%%%%%%%%%%%%%
%% some distributions
% continuous
\def\UnifDist{\text{\rm Unif}}
\def\BetaDist{\text{\rm Beta}}
\def\ExpDist{\text{\rm Exp}}
\def\GammaDist{\text{\rm Gamma}}
\def\NormalDist{\text{\rm Normal}}


% discrete
\def\BernDist{\text{\rm Bernoulli}}
\def\BinomDist{\text{\rm Binomial}}
\def\PoissonDist{\text{\rm Poisson}}
%%%%%%%%%%%%%%%%%%%%%%%%%%%%%%%%%%%%%%%%%%%%%%%%%%%%%%%%%%

%%%%%%%%%%%%%%%%%%%%%%%%%%%%%%%%%%%%%%%%%%%%%%%%%%%%%%%%%%
% Project-specific notation should go here
% (Because it's at the end of the file, it can overwrite anything that came before.)



%%%%%%%%%%%%%%%%%%%%%%%%%%%%%%%%%%%%%%%%%%%%%%%%%%%%%%%%%%



%%%%%%%%%%%%%%%%%%%%%%%%%%%%%%%%%%%%%%%%%%%%%%%%%%%%%%%%%%%%%%%%%%%%%%%%%%%%%%%%%

% your title/author/date information go here
\title{\todo} % replace with your title, a meaningful title
\author{Kenny Chiu} % replace with your name
\date{\today} % replace with your submission date


% start of document
\begin{document}

\maketitle

% summary section
% !TEX root = ../main.tex

% Summary section

\section{Summary}

\subsection{Context and background}

\citet{Lacotte:2020} study the theoretical performance of iterative Hessian sketch (IHS) for overdetermined least squares problems of the form
\[
\bfb^* = \argmin_{\bfb\in\bbR^d}\left\{f(\bfb) = \frac{1}{2}\|\bfX\bfb-\bfy\|^2\right\}
\]
where $\bfX\in\bbR^{n\times d}$, $n\geq d$, is a given full rank data matrix and $\bfy\in\bbR^n$ is a vector of observations. IHS is an iterative method based on random projections that is effective for ill-conditioned problems. Given step sizes $\{\alpha_t\}$ and momentum parameters $\{\beta_t\}$, the IHS solution is iteratively updated by 
\[
\bfb_{t+1} = \bfb_t - \alpha_t\bfH_t^{-1}\nabla f(\bfb_t)+\beta_t(\bfb_t-\bfb_{t-1}) \;.
\]
where the matrix $\bfH_t=\bfX^\T \bfS_t^\T \bfS_t\bfX$ is an approximation of the Hessian $\bfH=\bfX^\T\bfX$ given refreshed (i.i.d.) $m\times n$ sketching (random) matrices $\{\bfS_t\}$ with $m\ll n$. The theoretical performance of IHS with Gaussian sketches (i.e., where $(\bfS_t)_{ij}$ are \iid $N(0,m^{-1})$) has been studied, but IHS variants with other sketches have only been empirically studied. In their work, \citet{Lacotte:2020} draw on results from random matrix and free probability theory and show that the following sketches have faster (asymptotic) convergence rates compared to Gaussian sketches:
\begin{enumerate}

\item
truncated Haar sketch, where the rows of $\bfS_t$ are orthonormal. The orthogonality helps to prevent distortions in random projections but at the expense of requiring the Gram-Schmidt procedure, which has cost $O(nm^2)$ larger than the $O(nmd)$ cost when using Gaussian sketches.

\item
a version of the subsampled randomized Hadamard transform (SRHT), with $\bfS_t$ constructed from $\bfR_t=n^{-\frac{1}{2}}\bfB_t(\bfW_n)_t\bfD_t\bfP_t$ where $\bfB_t$ is a $n\times n$ diagonal matrix with \iid Bernoulli$\left(\frac{m}{n}\right)$ samples on the diagonal, $\bfD_t$ is a $n\times n$ diagonal matrix with uniformly sampled $\pm1$ on the diagonal, $\bfP_t$ is a $n\times n$ uniformly sampled row permutation matrix, and $(\bfW_n)_t$ is the $n\times n$ Walsh-Hadamard matrix defined recursively as
\[
\bfW_n =
\begin{bmatrix}
\bfW_\frac{n}{2} & \bfW_\frac{n}{2} \\
\bfW_\frac{n}{2} & -\bfW_\frac{n}{2}
\end{bmatrix}
\]
where $\bfW_1=1$. $\bfS_t$ is taken as $\bfR_t$ with the zeros rows removed (as selected by $\bfB_t$). Note that because of this subsampling, $\bfS_t$ is a $M\times n$ matrix with $\E[M]=m$. Sketching with SRHT only requires $O(nd\log M)$.

\end{enumerate}

\subsection{Main contributions}

The main contributions of \citet{Lacotte:2020} include several theoretical results that prescribe the (asymptotically) optimal value of the parameters for IHS with Haar or SRHT sketches, the corresponding convergence rates of IHS with these parameters, and closed form expression for the second inverse moment of SRHT sketches. These results are obtained based on asymptotic results from random matrix theory, in which it is assumed that the matrix dimensions satisfy the aspect ratios $\frac{d}{n}\rightarrow\gamma\in(0,1)$ and $\frac{m}{n}\rightarrow\xi\in(\gamma,1)$ as $n,d,m\rightarrow\infty$.
\\

The main results are Theorems~3.1 and 4.1. Theorem~3.1 says that for IHS with Haar sketches, the optimal rate $\rho_H$ at which the relative expected squared error decreases in each iteration is constant and proportional to the optimal rate $\rho_G$ of IHS with Gaussian sketches, and is given by
\[
\rho_H = \rho_G\cdot\frac{\xi(1-\xi)}{\gamma^2+\xi-2\xi\gamma} \;.
\]
The constant factor is less than 1 and therefore $\rho_H<\rho_G$, implying that IHS with Haar sketches converges at a faster rate than with Gaussian sketches.  Theorem~4.1 states a similar conclusion for IHS with SRHT sketches where the optimal rate $\rho_S$ is the same as that with Haar sketches, i.e., $\rho_S=\rho_H$, under an additional mild assumption on the initialization of the least squares problem which is necessary for drawing on known random matrix results. \todo optimal parameters

\subsection{Related literature}

\subsection{Limitations}

% mini-proposals section
% !TEX root = ../main.tex

% Mini-proposals section

\section{Mini-proposals}

% each mini-proposal gets its own subsection
\subsection[A sketched interior point algorithm for quantile regression]{Proposal 1: A sketched interior point algorithm for quantile regression} % enter your proposal title

Whereas linear regression fits a linear model on the conditional mean, quantile regression~\citep{Koenker:1978} fits a linear model on a conditional quantile. Quantile regression offers several advantages over linear regression, such as being able to model different quantiles (as opposed to only a mean), being free from assumptions regarding the parametric form of the response and homoscedasticity, and being transformation equivariant in its response~\citep{Rodriguez:2017}. Given a data matrix $\bfX\in\bbR^{n\times d}$, observations $\bfy\in\bbR^{n}$ and a quantile $\tau\in(0,1)$ of interest, the estimated parameters of the linear model are the solution to the optimization problem
\[
\min_{\bfb\in\bbR^d} \; \sum_{i=1}^n(y_i-\bfx_i^\T\bfb)\left(\tau - \mathbbm{1}[y_i-\bfx_i^\top\bfb < 0]\right) \;.
\]
The problem is non-differentiable as-is but can be optimized as a linear program. For large datasets, the conventional approach to solving the linear program is to use a constrained primal-dual interior point method~\citep{Portnoy:1997}. The interior point method involves iterative updates that are obtained as the solution to a linear system derived from a Newton step. The computational bottleneck in each update comes from  computing $\bfX^\T\bfW_t\bfX$ where $\bfW_t$ is a diagonal matrix that changes every iteration~\citep{Chen:2005}. This computation results in each iteration having a cost of $O(nd^2)$.
\\

In our proposed project, we consider the case where $d\ll n$ and propose a stochastic interior point algorithm that uses sketching matrices to reduce the computational cost of the iterative updates. Drawing on proven methods in the sketching literature~\cite{Pilanci:2017}, the idea is to incorporate a partial sketching step into the original algorithm where instead of computing $\bfX^\T\bfW_t\bfX$, we compute
\[
\bfX^\T\bfW_t^\frac{1}{2}\bfS_t^\T\bfS_t\bfW_t^\frac{1}{2}\bfX \;.
\]
The matrix $\bfS_t\in\bbR^{m\times n}$, $m\ll n$, is a random matrix regenerated every iteration that is introduced for reducing the dimension. For example, the subsampled randomized Hadamard transform allows the sketch $\bfS_t\bfW_t^\frac{1}{2}\bfX$ to be formed at a cost of $O(nd\log m)$~\citep{Lacotte:2020}, and the matrix product above can then be computed at a cost of $O(md^2)$. While the sketched solution will only be an approximation to the original solution, recent work on the convergence of sketched solutions in other optimization problems show promising theoretical and empirical results~\citep[e.g.,][]{Pilanci:2017,Derezinski:2021,Lacotte:2021}. We also note that \citet{Yang:2013} had previously proposed a stochastic algorithm for quantile regression. However, their method differs greatly from ours in that they construct a random preconditioning matrix before using standard methods to solve the optimization problem on the conditioned data matrix.
\\

The main contributions of this project would be as follows:
\begin{enumerate}
\item
A sketched interior point algorithm for optimizing quantile regression problems that is expected to be faster than standard methods currently used in practice.
\item
A theoretical analysis of the proposed sketched interior point algorithm that provides convergence guarantees.
\item
An empirical comparison of quantile regression models fitted on large datasets obtained from the proposed sketched interior point algorithm and other existing methods, such as the standard interior point method~\citep{Portnoy:1997} (implemented in R), the stochastic method by \citet{Yang:2013}, a more modern iteration of the interior point method~\citep{Zhao:2020}, and a modern quantile regression algorithm based on smoothing~\citep{He:2021} (also implemented in R).
\item
An implemention of the sketch interior point algorithm, e.g., in R, if found to have practical advantages over the existing algorithms.
\end{enumerate}

The main challenge of this project would be the theoretical analysis of the sketched interior point algorithm. The most feasible analysis approach would likely be following that of \citet{Pilanci:2017} for interior point methods and partial sketches, which would provide a worst-case result about the number of iterations needed to obtain a solution within a desired tolerance. The effect of the sketching matrix may also be of interest, but an analysis approach similar to the asymptotic approach of \citet{Lacotte:2020} would likely be necessary. However, adapting their approach for least squares to that of quantile regression is not straightforward and would likely be more suited for a follow-up project.
\\

Following the completion of this project, there are multiple directions of future work that may be of interest:
\begin{enumerate}
\item
The proposed sketched interior point algorithm would be useful for the $d\ll n$ case but not for the $n\ll d$ case. For the latter, a sketch-based method would likely still be possible but would need to use sketches differently, e.g., directly sketching the data matrix as \citet{Pham:2015} did for LASSO or sketching both the data matrix and the observations as in classical least-squares sketch (although sketching both has been shown to be suboptimal~\citep{Pilanci:2016}).
\item
Exploration of applications that may benefit from a faster quantile regression or interior point algorithm, e.g., composite quantile regression~\citep{Zou:2008} for high-dimensional regression or applications of quadratic progamming.
\item
Investigation of sketched quantile regression algorithms based on smoothing. These algorithms approximate the original optimization problem by a differentiable one and therefore sketching should directly follow from the work of \citet{Pilanci:2017}. Given the more standard setup, it is likely easier to adapt the approach of \citet{Lacotte:2020} to these problems than it is to the interior point algorithms for studying the effect of specific sketching matrices in sketched quantile regression.
\end{enumerate}


\newpage


% each mini-proposal gets its own subsection
\subsection{Proposal 2: MY OTHER PROPOSAL TITLE} % enter your proposal title

% ...

% project report section
% !TEX root = ../main.tex

% Project report section

\section{Project report}

\subsection{Introduction}

Ridge regression is a special case of regularized least squares where the penalty function is chosen to be the $\ell_2$-norm of the model parameters. Given data matrix $\bfX\in\bbR^{n\times d}$, observations $\bfy\in\bbR^n$ and a regularization parameter $\lambda>0$, ridge regression obtains estimates of the parameters as the solution to the optimization problem
\[
\bfb^* = \argmin_{\bfb\in\bbR^d} \frac{1}{2}\|\bfX\bfb-\bfy\|^2 + \frac{\lambda}{2}\|\bfb\|^2 \;.
\]
While ridge regression can be motivated as a method for reducing overfitting in ordinary least squares (OLS), it also has its computational and analytical benefits over OLS. When $\bfX$ does not have full column rank (e.g., when $n< d$), then $\bfX^\T\bfX$ is singular and the OLS solution is non-unique. When $\bfX$ is full rank but ill-conditioned, then small changes in $\bfX$ lead to large changes in $(\bfX^\T\bfX)^{-1}$ and consequently in the OLS solution. Ridge regression addresses both of these issues by minimizing the variance and mean squared error at the cost of introducing a small bias~\citep{Chowdhury:2018}. The ridge regression solution is unique and is given by
\[
\bfb^* = \left(\bfX^\T\bfX + \lambda\bfI_d\right)^{-1}\bfX^\T\bfy \;.
\]
In this report, we analyze the theoretical properties of a partial Newton sketch algorithm~\citep{Pilanci:2017} as an iterative solver for the ridge regression problem. In particular, we attempt to derive an optimal convergence rate and an optimal step size following the approach of \citet{Lacotte:2020} for iterative Hessian sketch with OLS using asymptotic results from random matrix theory and free probability. We show that while ridge regression can be considered a simple extension to OLS, extending the analysis approach of \citet{Lacotte:2020} to partial Newton sketch is not trivial. \todo
\\

This report is organized as follows: Section~\ref{sec:background} provides background about sketching and describes the OLS results by \citet{Lacotte:2020} that we aim to extend to ridge regression; Section~\ref{sec:literature} highlights relevant work in the literature; Section~\ref{sec:theory} discusses our attempts to analyze Newton sketch for ridge regression and the key differences from OLS that makes the problem challenging; Section~\ref{sec:empirical} describes \todo; and Section~\ref{sec:conclusion} summarizes our findings and concludes this report.

\subsection{Background} \label{sec:background}

\subsubsection{Sketching and Newton sketch}

\subsubsection{Optimal convergence of iterative Hessian sketch}

\subsubsection{Notation}

\subsection{Related work} \label{sec:literature}

\subsection{Newton sketch for ridge regression} \label{sec:theory}

\todo
Gradient:
\[
\nabla f(\bfb_t) = \left(\bfX^\T\bfX+\lambda\bfI_d\right)\bfb_t - \bfX^\T\bfy
\]

Hessian and partial Newton sketch (as considered by \citet{Chowdhury:2018}, \citet{Wang:2017}):
\begin{align*}
\bfH &= \bfX^\T\bfX+\lambda\bfI_d \\
\bfH_t &= \bfX^\T\bfS_t^\T\bfS_t\bfX + \lambda\bfI_d
\end{align*}

No momentum
\begin{align*}
\bfb_{t+1} &= \bfb_t - \alpha_t\bfH_t^{-1}\nabla f(\bfb_t) \\
&= \bfb_t - \alpha_t\left(\bfX^\T\bfS_t^\T\bfS_t\bfX + \lambda\bfI_d\right)^{-1}\left(\left(\bfX^\T\bfX+\lambda\bfI_d\right)\bfb_t-\bfX^\T\bfy\right)
\end{align*}

\subsection{Analysis, assumptions and challenges}

\todo

Assumption~1: $\bfX$ has full column rank.

Define $\Delta_t=\bfU^\T\bfX\left(\bfb_t-\bfb^*\right)$.

Assumption~2: $\Delta_0$ is random and $\E\left[\Delta_0\Delta_0^\T\right]=\frac{\bfI_d}{d}$


Using the fact that the ridge regression solution satisfies the equation
\[
(\bfX^\T\bfX + \lambda\bfI_d)\bfb^* = \bfX^\T\bfY \;,
\]
the \todo iteration can be rewritten as
\begin{align*}
\bfb_{t+1} &= \bfb_t - \alpha_t\left(\bfX^\T\bfS_t^\T\bfS_t\bfX + \lambda\bfI_d\right)^{-1}\left(\left(\bfX^\T\bfX+\lambda\bfI_d\right)\bfb_t-\left(\bfX^\T\bfX+\lambda\bfI_d\right)\bfb^*\right) \\
&= \bfb_t - \alpha_t\left(\bfX^\T\bfS_t^\T\bfS_t\bfX + \lambda\bfI_d\right)^{-1}\left(\bfX^\T\bfX+\lambda\bfI_d\right)\left(\bfb_t-\bfb^*\right) \;.
\end{align*}
Let $\bfX = \bfU\Sigma\bfV^\T$ be the thin singular value decomposition of $\bfX$. By Assumption~1, $\bfU$ is a $n\times d$ semi-orthogonal matrix, $\bfV$ is a $d\times d$ orthogonal matrix, and $\Sigma$ is a $d\times d$ diagonal matrix with the non-zero singular values of $\bfX$ on the diagonal. Then we can write
\begin{align*}
\bfX^\T\bfX+\lambda\bfI_d &= \bfV\Sigma^2\bfV^\T + \lambda \bfV\Sigma\Sigma^{-2}\Sigma\bfV^\T \\
&= \bfV\Sigma\left(\bfI_d + \lambda\Sigma^{-2}\right)\Sigma\bfV^\T \\
\left(\bfX^\T\bfS_t^\T\bfS_t\bfX + \lambda\bfI_d\right)^{-1} &= \left(\bfV\Sigma^\T\bfU^\T\bfS_t^\T\bfS_t\bfU\Sigma\bfV^\T + \lambda\bfV\Sigma\Sigma^{-2}\Sigma\bfV^\T\right)^{-1} \\
&= \bfV\Sigma^{-1}\left(\bfU^\T\bfS_t^\T\bfS_t\bfU + \lambda\Sigma^{-2}\right)^{-1}\Sigma^{-1}\bfV^\T
\end{align*}
The \todo iteration then becomes
\[
\bfb_{t+1} = \bfb_t - \alpha_t\bfV\Sigma^{-1}\left(\bfU^\T\bfS_t^\T\bfS_t\bfU + \lambda\Sigma^{-2}\right)^{-1}\left(\bfI_d + \lambda\Sigma^{-2} \right)\Sigma\bfV^\T\left(\bfb_t-\bfb^*\right) \;.
\]
Multiplying both sides by $\bfU^\T\bfX$ then gives
\begin{align*}
\bfU^\T\bfX\bfb_{t+1} &= \bfU^\T\bfX\bfb_t - \alpha_t\bfU^\T\bfX\bfV\Sigma^{-1}\left(\bfU^\T\bfS_t^\T\bfS_t\bfU + \lambda\Sigma^{-2}\right)^{-1}\left(\bfI_d + \lambda\Sigma^{-2} \right)\Sigma\bfV^\T\left(\bfb_t-\bfb^*\right) \\
&= \bfU^\T\bfX\bfb_t - \alpha_t\left(\bfU^\T\bfS_t^\T\bfS_t\bfU + \lambda\Sigma^{-2}\right)^{-1}\left(\bfI_d + \lambda\Sigma^{-2} \right)\Sigma\bfV^\T\left(\bfb_t-\bfb^*\right)
\end{align*}
and subtracting both sides by $\bfU^\T\bfX\bfb^*$ gives
\begin{align*}
\bfU^\T\bfX(\bfb_{t+1}-\bfb^*) &= \bfU^\T\bfX(\bfb_t-\bfb^*) - \alpha_t\left(\bfU^\T\bfS_t^\T\bfS_t\bfU + \lambda\Sigma^{-2}\right)^{-1}\left(\bfI_d + \lambda\Sigma^{-2} \right)\Sigma\bfV^\T\left(\bfb_t-\bfb^*\right) \\
&= \bfU^\T\bfX(\bfb_t-\bfb^*) - \alpha_t\left(\bfU^\T\bfS_t^\T\bfS_t\bfU + \lambda\Sigma^{-2}\right)^{-1}\left(\bfI_d + \lambda\Sigma^{-2} \right)\bfU^\T\bfX\left(\bfb_t-\bfb^*\right) \\
&= \left(\bfI_d - \alpha_t\left(\bfU^\T\bfS_t^\T\bfS_t\bfU + \lambda\Sigma^{-2}\right)^{-1}\left(\bfI_d + \lambda\Sigma^{-2}\right)\right)\bfU^\T\bfX\left(\bfb_t-\bfb^*\right) \;.
\end{align*}
Therefore, by definition,
\[
\Delta_{t+1} = \left(\bfI_d - \alpha_t\left(\bfU^\T\bfS_t^\T\bfS_t\bfU + \lambda\Sigma^{-2}\right)^{-1}\left(\bfI_d + \lambda\Sigma^{-2}\right)\right)\Delta_t
\]
and
\[
\|\Delta_{t+1}\|^2 = \Delta_t^\T\left(\bfI_d - \alpha_t\left(\bfU^\T\bfS_t^\T\bfS_t\bfU + \lambda\Sigma^{-2}\right)^{-1}\left(\bfI_d + \lambda\Sigma^{-2}\right)\right)^2\Delta_t \;.
\]
Taking the expectation with respect to $\bfS_t$, we get
\[
\E\left[\|\Delta_{t+1}\|^2\right] = \Delta_t^\T\E\left[\left(\bfI_d - \alpha_t\left(\bfU^\T\bfS_t^\T\bfS_t\bfU + \lambda\Sigma^{-2}\right)^{-1}\left(\bfI_d + \lambda\Sigma^{-2}\right)\right)^2\right]\Delta_t \;.
\]
\todo Theorem~3.1 avoids Assumption~2 by exploiting the rotational invariance of $\bfU^\T\bfS_t^\T\bfS_t\bfU$ \todo.

Take $\bfQ_t=\bfI_d - \alpha_t\left(\bfU^\T\bfS_t^\T\bfS_t\bfU + \lambda\Sigma^{-2}\right)^{-1}\left(\bfI_d + \lambda\Sigma^{-2}\right)$. \todo satisfies conditions (limiting spectral distribution as $n\rightarrow\infty$ and others?). Then we have $\Delta_{t+1}=\bfQ_t\Delta_t$ and so
\begin{align*}
\E\left[\|\Delta_{t+1}\|^2\right] &= \mathrm{trace}\left(\E\left[\Delta_0^\T\bfQ_0\ldots \bfQ_{t-1}\bfQ_{t-1}\ldots\bfQ_0\Delta_0\right]\right) \\
&= \mathrm{trace}\left(\E\left[\bfQ_0\ldots \bfQ_{t-1}\bfQ_{t-1}\ldots\bfQ_0\Delta_0\Delta_0^\T\right]\right) \\
&= \mathrm{trace}\left(\E\left[\bfQ_1\ldots \bfQ_{t-1}\bfQ_{t-1}\ldots\bfQ_1\bfQ_0^2\right]\E\left[\Delta_0\Delta_0^\T\right]\right) \\
&= \frac{1}{d}\mathrm{trace}\left(\E\left[\bfQ_1\ldots \bfQ_{t-1}\bfQ_{t-1}\ldots\bfQ_1\bfQ_0^2\right]\right)
\end{align*}
using the independence of $\Delta_0$ and $\bfQ_i$ and using Assumption~2. Then taking the limit in $n$ and recursively applying the fact that $\bfQ_0^2$ is asymptotically free from $\bfQ_{t-1}\ldots\bfQ_1$ \todo, we get
\begin{align*}
\lim_{n\rightarrow\infty}\E\left[\|\Delta_{t+1}\|^2\right] &= \lim_{n\rightarrow\infty}\frac{1}{d}\mathrm{trace}\left(\E\left[\bfQ_1\ldots \bfQ_{t-1}\bfQ_{t-1}\ldots\bfQ_1\bfQ_0^2\right]\right) \\
&= \lim_{n\rightarrow\infty}\frac{1}{d}\mathrm{trace}\left(\E\left[\bfQ_0^2\right]\right)\lim_{n\rightarrow\infty}\frac{1}{d}\mathrm{trace}\left(\E\left[\bfQ_2\ldots \bfQ_{t-1}\bfQ_{t-1}\ldots\bfQ_2\bfQ_1^2\right]\right) \\
&= \prod_{i=0}^{t-1}\lim_{n\rightarrow\infty}\frac{1}{d}\mathrm{trace}\left(\E\left[\bfQ_i^2\right]\right)
\end{align*}
\todo The expectation of $\bfQ_i$ is given by
\begin{align*}
\E\left[\bfQ_i^2\right] &= \bfI_d - \alpha_i\E\left[\left(\bfU^\T\bfS_i^\T\bfS_i\bfU + \lambda\Sigma^{-2}\right)^{-1}\right]\left(\bfI_d + \lambda\Sigma^{-2}\right) \\
&\quad - \alpha_i\left(\bfI_d + \lambda\Sigma^{-2}\right)\E\left[\left(\bfU^\T\bfS_i^\T\bfS_i\bfU + \lambda\Sigma^{-2}\right)^{-1}\right] \\
&\quad + \alpha_i^2\left(\bfI_d + \lambda\Sigma^{-2}\right)\E\left[\left(\bfU^\T\bfS_i^\T\bfS_i\bfU + \lambda\Sigma^{-2}\right)^{-2}\right]\left(\bfI_d + \lambda\Sigma^{-2}\right)
\end{align*}
and the normalized limiting trace is given by
\begin{align*}
\lim_{n\rightarrow\infty}\frac{1}{d}\mathrm{trace}\left(\E\left[\bfQ_i^2\right]\right) &= 1 - \frac{2\alpha_i}{d}\lim_{n\rightarrow\infty}\mathrm{trace}\left(\E\left[\left(\bfU^\T\bfS_i^\T\bfS_i\bfU + \lambda\Sigma^{-2}\right)^{-1}\right]\left(\bfI_d + \lambda\Sigma^{-2}\right)\right) \\
&\quad + \frac{\alpha_i^2}{d}\lim_{n\rightarrow\infty}\mathrm{trace}\left(\E\left[\left(\bfU^\T\bfS_i^\T\bfS_i\bfU + \lambda\Sigma^{-2}\right)^{-2}\right]\left(\bfI_d + \lambda\Sigma^{-2}\right)^2\right)
\end{align*}

\subsection{Empirical experiments} \label{sec:empirical}


\subsection{Conclusion} \label{sec:conclusion}


\newpage


\section{Comment}

I would like to briefly comment on the iterative Hessian sketch/Newton sketch literature. While the body of works that focus on these algorithms is fairly substantial with many new developments in recent years, there appears to be only one or two research groups driving this portion of the literature and reinforcing their position with many self-citations. Many of these citations are to unpublished papers or concurrent papers. To make matters more confusing, there are instances of cyclical citations where one paper references an old version of another paper, and the new version of the other paper references the paper that references its old version (e.g., \citep{Lacotte:2020} and \citep{Lacotte:2020b}). This made tracking the lineage of the developments in this literature a bit of a nightmare.


\newpage


\bibliographystyle{plainnat}
\bibliography{../../references/qp}

\end{document}
